\documentclass[12pt,letterpaper, margin=1in]{article}
%%% Import Packages %%%
\usepackage[margin=1in]{geometry}
\usepackage{adjustbox}
\usepackage[utf8]{inputenc}
\usepackage{amsmath}
\usepackage{amsfonts}
\usepackage{amssymb}
\usepackage{fancyhdr}
\usepackage{graphicx}
\usepackage{booktabs}
%%%

%%% Header and Footer Text %%%
\pagestyle{fancy}
\fancyhf{}
\fancyhead[LE,RO]{e-store}
\fancyhead[RE,LO]{Software Quality Assurance Plan}
\fancyfoot[CE,CO]{\leftmark}
\fancyfoot[LE,RO]{\thepage}

\renewcommand{\headrulewidth}{2pt}
\renewcommand{\footrulewidth}{1pt}
 %%% 
 
\begin{document}

\title{\textbf{e-store}\\Software Quality Assurance Plan\\{\normalsize Version 0.4.0, 11 March 2016}\\[25pt]}
\author{
	\textbf{Project Team}\\
	Ebarle, Roselle M.\\
	Maglasang, Catherine M.\\
	Yee, Mar Rynner\\
	Esin, Dexter D.\\	[25pt]
	%%%%
	\textbf{Project Manager}\\
	Ebarle, Roselle M.\\[25pt]
	%%%%
	\textbf{Senior Manager}\\
	Maglasang, Catherine M.\\[25pt]
	%%%%
	\textbf{Advisor}\\
	Prof. Orven Llantos Ebarle\\[25pt]
	%%%%
}
\clearpage\maketitle
\thispagestyle{empty}

\newpage

\begin{abstract}
This document is about the Software Quality Assurance Plan $(SQAP)$ of the system, e-store which is an e-commerce platform for online merchants. It provides the power to grow your web business, reach more customers and sell more products and services. It enables businesses to experience an integrated workflow for their business: Sales, Inventory and Order Management and Customer Service under one platform.
\end{abstract}

\newpage 
\tableofcontents
\newpage 

\section*{Document Status Sheet}
%%
\begin{table}[ht]
\centering
\begin{adjustbox}{width=1\textwidth,center=\textwidth}
\small
\begin{tabular}{| l | l |}
\hline
\textbf{Document Title} & \textbf{Software Quality Assurance Plan} \\
\hline
\textbf{Document Identification} & e-store/Documents/Management/SQAP/0.5.0 \\ 
\textbf{Author(s)} & Ebarle, Maglasang, Esin, Yee \\
\textbf{Version} & 0.4.0 \\ 
\textbf{Document Status} & draft \\
\hline
\end{tabular}
\end{adjustbox}
\end{table}
%%
\begin{table}[ht]
\centering
\begin{adjustbox}{width=1\textwidth,center=\textwidth}
\small
\begin{tabular}{| l | l | l | l |}
\hline
\textbf{Version} & \textbf{Date} & \textbf{Author(s)} & \textbf{Summary} \\
\hline
0.0.0 & 05-02-2016 & R. Ebarle, C, Maglasang, M. Yee, D. Esin & Document Start \\ 
0.1.0 & 13-02-2016 & R. Ebarle, C, Maglasang, M. Yee, D. Esin & Edited Chapter 1, Started Chapter 2 \\ 
0.2.0 & 13-02-2016 & R. Ebarle, C, Maglasang, M. Yee, D. Esin & Added Chapter 3 \\
0.3.0 & 13-02-2016 & R. Ebarle, C, Maglasang, M. Yee, D. Esin & Added Chapter 4 \\ 
0.4.0 & 11-03-2016 & R. Ebarle, C, Maglasang, M. Yee, D. Esin & Added Chapter 5 \\ 
\hline
\end{tabular}
\end{adjustbox}
\end{table}
%%

\newpage
\section{Introduction}
\subsection{Purpose}
The purpose of this plan is to define the e-store Software Quality Assurance (SQA) organization, SQA tasks and responsibilities; provide reference documents and guidelines to perform the SQA activities; provide the standards, practices and conventions used in carrying out SQA activities; and provide the tools, techniques, and methodologies to support SQA activities, and SQA reporting.
	
\subsection{Scope}
This plan establishes the SQA activities performed throughout the life cycle of the e-store project.

This plan shall implement a project that follows the RESTful architectural style. The project shall be developed using the Flask microframework. There will be a clear separation of concerns between the client and the server for easy maintenance and scalability.

\newpage

\subsection{List of Definitions}
\begin{table}[ht]
\centering
\begin{adjustbox}{width=1\textwidth,center=\textwidth}
\small
\begin{tabular}{| l | l |}
\hline
\textbf{Term} & \textbf{Definition} \\
\hline
ATDD & Acceptance Test Driven Development \\
TDD & Test Driven Development \\
BDD & Behavior-Driven Development \\
SQA & Software Quality Assurance \\
UML & Unified Modeling Language \\
ERD & Entity Relationship Diagram \\
MSU-IIT & Mindanao State University - Iligan Institute of Technology \\
REST & Representational State Transfer \\
FLASK & Web Framework \\
Python & Programming Language \\
SCS & School of Computer Studies \\ 
Sales Inventory & The list of items such as the goods that are in stock \\
E-commerce & The buying and selling of goods over an electronic network, primarily the internet \\
Customer & The person who transacts in the store page of the business \\
Admin & The owner of the products sold in the store. \\
Product & The items being sold in the website \\
Cart & The list of items the customer is going to buy \\
Checkout & The process in which the customer is going to buy and pay the items inside the cart \\
Gherkin & Business Readable, Domain Specific Language that lets you describe software’s behaviour without detailing how that behaviour is implemented. \\
QAM & Quality Assurance Manager \\
SQAP & Software Quality Assurance Plan \\
SQMP & Software Quality Management Plan \\
PM & Project Manager \\
CM & Configuration Manager \\
AD & Architectural Design \\
DD & Detailed Design \\ 
CI & Configuration Items \\
UML & Unified Modeling Language \\
\hline
\end{tabular}
\end{adjustbox}
\caption{List of Definitions} 
\end{table}

\newpage

\subsection{List of References}
\begin{itemize}
\item $[SQAP]$ Software Quality Assurance Plan, SPINGRID team, TU/e, 0.1.3, June 2006
\item Saleh H. (2013). Javascript Unit Testing.  Packt Publishing
\item Osmani A., (2012). Javascript Learning Design Patterns
\item Zlobin G., (2013). Learning Python Design Patterns
\item Sale D., (2014). Testing Python
\item IEEE Standard for Software Quality Assurance Processes, IEEE Std 730-2014
\item Clean Code Cheat Sheet
\item Test-Driven Development, Dr. Christoph Steindl, Senior IT Architect and Method Exponent, Certified ScrumMaster
\item Best Practices, Development Methodologies, and the Zen of Python, Valentin Haenel
\item Test-Driven Development, Gary Brown
\item Detailed Design, (2006). Parametric Technology Corporation (PTC)
\item Configuration Items, $http://www.chambers.com.au/glossary/configuration_item.php$
\item $http://flask.pocoo.org/docs/0.10/styleguide/$
\item $http://explore-flask.readthedocs.org/en/latest/conventions.html$
\end{itemize}

\newpage
\section{Management}
This section describes each major element of the organization that influences the quality of the software.

\subsection{Organization}

The team shall follow the agile approach, and adhere to the scrum approach in development. The team shall consist of the Scrum Master, Product Owner and the Development Team. Throughout each iteration, SQA activities should be headed by the Scrum Master who shall also serve as the Quality Assurance Manager. The team shall follow the Behaviour-driven approach in development as an extension to the Test-driven development approach. Prior to coding any functionality, the individual responsible for the feature shall create just enough acceptance tests, unit tests and code to pass the tests. 

\subsection{Tasks}
The SQA team’s main task is to check whether the procedures are followed and that standards are handled correctly as defined in the [SQAP]. Additionally, the SQA team inspects whether all group members fulfill their tasks according to the parts of the [SQAP] applying to their specific tasks. \\

Besides the described main task, the SQA team has to check the consistency and coherence between documents.

\subsection{Responsibilities}
The responsibility of Quality Assurance shall be vested in all of the members of the development team. Each one shall serve as a tester and developer at the same time. However, a Software Quality Assurance Manager (QAM) shall be the one to oversee that the BDD approach is followed, and that tests cover 100 percent coverage throughout the system, in order to avoid any bleeds or regression. The agile team shall be self-organizing individuals and take full responsibility in the feature or story assigned to them. The team shall not only be concerned with the product quality but also with the process quality and relationship between them. Should there be any major problems, the QAM shall take over and plan as needed. 

\newpage
\section{Documentation}
The documents to be delivered in the specific phases of the project will be based in Chapter 4. Document standards are described in the same chapter.

\newpage
\section{Standards, Practices, Conventions And Metrics}
\subsection{Documentation Standards}
Documentations may be in the form of a test, a docstring, or any formal document. If possible, the code should serve as enough documentation for the system. Throughout the project, PEP 8 style guide convention shall be used. 

\subsubsection{PEP 8 basically commands developers the following practices: }
\begin{itemize}
\item Indentation: Indent with 4 real spaces (no tabs)
\item Maximum line lenght: 79 characters with a soft limit for 84 if absolutely necessary. Try to avoid too nested code by cleverly placing break, continue and return statements.
\item Continuing long statements: To continue a statement you can use backslashes in which case you should align the next line with the last dot or equal sign, or indent four spaces.
\end{itemize}

\subsubsection{Docstrings}
All docstrings shall be formatted with reStructuredText as understood by Sphinx. Depending on the number of lines in the docstring, they are laid out differently. If it’s just one line, the closing triple quote is on the same line as the opening, otherwise the text is on the same line as the opening quote and the triple quote that closes the string on its own line:

\subsubsection{Comments}
Rules for comments are similar to docstrings. Both shall be formatted with reStructuredText. If a comment is used to document an attribute, put a colon after the opening pound sign $(\#)$. 

\subsection{Design Standards}

The system shall follow the restful-architectural style. The API backend shall be built with Flask, while the frontend (running in a different port) shall be built with AngularJS. The following table describes the API model of the project: 

\begin{table}[ht]
\resizebox{\textwidth}{!}{%
\begin{tabular}{llll}
\hline
\textbf{Resource}  			& \textbf{URI}      & \textbf{HTTP Method} & \textbf{Description}							\\ \hline
Inventory Module  \\ \hline
Item      				& /api/v1/items/ & GET & Retrieve all items       			\\
          					& /api/v1/items/:id & GET & Retrieve single item 			\\
          					& /api/v1/items/:id & PUT & Update item 						\\ 
          					& /api/v1/items/ & POST & Create new item 					\\ \hline
Type						& /api/v1/types/ & GET & Retrieve all item types 			\\
							& /api/v1/types/:id & GET & Retrieve all types \\ 
							& /api/v1/types/:id & PUT & Update type \\ 
							& /api/v1/types/ & POST & Create new type \\ \hline
Attribute				& /api/v1/types/:id/attributes/ & GET & Retrieve all attributes under the type id \\ 
							& /api/v1/types/:id/attributes/:id/ & GET & Retrieve single attribute under the type id \\
							& /api/v1/types/:id/attributes/ & POST & Create new attribute under the type id \\
							& /api/v1/types/:id/attributes/:id/ & PUT & Update attribute details under the type id \\ \hline
AttributeValue		& /api/v1/items/:id/attributes/ & GET & Retrieve all type-attribute pair values for each item assigned to a particular type \\ 
							& /api/v1/items/:id/attributes/:id/ & GET & Retrieve a single type-attribute pair value for an item \\ 
							& /api/v1/items/:id/attributes/:id/ & PUT & Update the type-attribute pair value \\ 
							& /api/v1/items/:id/attributes/ & POST & Create new attribute value based on the type assigned to an item \\ \hline
Image 					& /api/v1/items/:id/images/ & GET & Retrieve all images for a single item\\
							& /api/v1/items/:id/images/:id/ & GET & Retrieve single image for an item \\ 
							& /api/v1/items/:id/images/:id/ & PUT & Update image for an item \\
							& /api/v1/items/:id/images/ & POST & Create new image for an item \\ \hline
Supplier				& /api/v1/suppliers/ & GET & Retrieve all suppliers \\ 
							& /api/v1/suppliers/:id/ & GET & Retrieve single supplier \\ 
							& /api/v1/suppliers/:id/ & PUT & Update a supplier \\ 
							& /api/v1/suppliers/	& POST & Create new supplier \\ \hline
Cart \& POS Module  \\ \hline
Cart 						& /api/v1/carts/			& POST & Create new cart instance  \\ 
							& /api/v1/carts/:id/ 	& PUT &  Update Cart  \\ \hline
CartItem				& /api/v1/carts/:id/items/ & GET & Retrieve all cart items \\ 
							& /api/v1/carts/:id/items/:id/ & GET & Retrieve single cart item \\
							& /api/v1/carts/:id/items/:id/ & PUT & Update single cart item \\
							& /api/v1/carts/:id/items/ & POST & Add an item to cart \\ \hline
Order					& /api/v1/orders/ & GET & Retrieve all orders \\ 
							& /api/v1/orders/:id/ & GET & Retrieve single order \\ 
							& /api/v1/orders/:id/ & PUT & Update single order \\ 
							& /api/v1/orders/ & POST & Create new order \\ \hline
OrderItem				& /api/v1/orders/:id/items/ & GET & Retrieve all order items \\ 
							& /api/v1/orders/:id/items/:id/ & GET & Retrieve single order item \\ 
							& /api/v1/orders/:id/items/:id/ & PUT & Update single order item \\
							& /api/v1/orders/:id/items/ & POST & Add new item in order \\ \hline
Wishlist 				& /api/v1/wishlists/ & GET & Retrieve all created wishlists \\ 
							& /api/v1/wishlists/:id/ & GET & Retrieve single wishlist \\ 
							& /api/v1/wishlists/:id/ & PUT & Update wishlist \\ 
							& /api/v1/wishlists/ & POST & Create new wishlist \\ \hline
WishlistItem			& /api/v1/wishlists/:id/items/ & GET & Retrieve all items under a single wishlist \\ 
							& /api/v1/wishlists/:id/items/:id/ & GET & Retrieve single wishlist item \\ 
							& /api/v1/wishlists/:id/items/:id/ & PUT & Update wishlist item \\
							& /api/v1/wishlists/:id/items/ & POST & Add new item in wishlist  \\ \hline
User 					& /api/v1/users/ & GET & Retrieve all users \\ 
							& /api/v1/users/:id/ & GET & Retrieve single user \\
							& /api/v1/users/:id/ & PUT & Update user \\ 
							& /api/v1/users/ & POST & Create new user \\  \hline							
Group 					& /api/v1/groups/ & GET & Retrieve all groups \\ 
							& /api/v1/groups/:id/ & GET & Retrieve single group \\
							& /api/v1/groups/:id/ & PUT & Update group \\ 
							& /api/v1/groups/ & POST & Create new group \\  \hline		
							& /api/v1/groups/:id/users/ & GET & Retrieve all users under a group \\ 
							& /api/v1/groups/:id/users/:id/ & GET & Retrieve a user under a group \\
							& /api/v1/groups/:id/users/:id/ & PUT & Update a user in a group (e.g. Permission) \\
							& /api/v1/groups/:id/users/ & POST & Add a user to a group \\ \hline											
Customer 			& /api/v1/site/:id/customers/ & GET & Retrieve all customers in a site \\ 
							& /api/v1/site/:id/customers/:id/ & GET & Retrieve single customer in a site \\
							& /api/v1/site/:id/customers/:id/ & PUT & Update customer in a site \\ 
							& /api/v1/site/:id/customers/ & POST & Register new customer in site or business \\ 				
\end{tabular}
}
\caption{REST API Model}
\label{API Model}
\end{table}

\subsection{Coding Standards}

\subsection{Comment Standards}

\subsection{Testing Standards}

\subsection{Metrics}

\subsection{Compliance Monitoring}

\section{Review}

\section{Test}

\section{Problem reporting and corrective actions}
\section{Tools, techniques and methods}
\section{Code Control}
\section{Media Control}
\section{Supplier Control}
\section{Records collection, maintenance and retention}

\section{Training}
The project requires sufficient skill in Python, Flask, and front end technologies like Angular, jQuery and Ajax. The learning curve throughout the project development has been steep and required training from the Advisor and co-team members. 

\section{Risk management}

\subsection{Categories of risks}
The following are categories of risks that are relevant to the project: 

\subsubsection{Risks with respect to the work to be done}
\textbf{1. Miscommunication} \\
\textit{Probability:} High \\
\textit{Prevention:} Daily stand ups or quick huddle should be done by the team on a regular basis. Major weekly meetings are done to keep address pressing issues in development. Team members should not hesitate to ask and re ask questions if things are unclear in order to avoid bottlenecks in the progress of the system. With regards to the customer, bi-monthly face-to-face meetups should be done to update and keep track of progress. If any confusions arise, the team may opt to use other communication mediums like phone calls, or emails to clear up problems. \\
\textit{Correction:} When it becomes clear that miscommunication is causing problems, the team members involved and the customer are gathered in a meeting to clear things up. \\
\textit{Impact:} High \\

\subsubsection{Risks with respect to the customer}


\end{document}